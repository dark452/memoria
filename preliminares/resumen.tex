\chapter* {Resumen}

\addcontentsline{toc}{chapter}{Resumen}
	
\noindent El presente trabajo de t�tulo aborda la evaluaci�n del sistema de administraci�n de base de datos Oracle 11g, 
con el objetivo de investigar las capacidades que este posee. Para ello se dise�ar� un caso de estudio que luego ser� 
implementado en una base de datos relacional, base de datos objeto relacional y el manejo de datos semi estructurados como XML, 
con el prop�sito de obtener como resultado un documento  que indique las ventajas y desventajas de  cada implementaci�n considerando 
criterios como facilidad de modelamiento y facilidad de aprendizaje por parte del desarrollador.
	
\noindent En el presente informe se hablar� de los distintos conceptos
nombrados como lo son bases de datos relacionales, bases de datos objeto
relacional y datos semi estructurados como XML. Dentro de la introducci�n se justificar� el prop�sito de la realizaci�n de esta investigaci�n, los objetivos m�s relevantes, 
una descripci�n detallada del trabajo a realizar, junto a los resultados esperados de dicho trabajo, 
especificando la metodolog�a de investigaci�n. Los recursos tanto como hardware,
software y personas involucradas. Se detallar� a fondo un marco te�rico de cada
tema antes mencionado, adem�s se mencionar� brevemente algunos
motores de bases de datos con capacidades objeto relacional y sus ejemplos
correspondientes. Por �ltimo dentro del marco te�rico se especificar� en profundidad
las capacidades que posee el motor Oracle, detallando tanto teor�a como ejemplos
de implementaci�n. 

\noindent En el caso de estudio se presentar� el problema a desarrollar
con sus respectivos diagramas de entidad relacionamiento y UML, tambi�n se propondr� un
algoritmo para transformar desde el modelo relacional al objeto
relacional y se especificar� cada paso de la implementaci�n,
las dificultades a la que se vio enfrentada la implementaci�n del modelo objeto
relacional y las soluciones propuestas. Se incluir� un nuevo requerimiento al
caso de estudio para implementar los datos semi estructurados en el modelo
objeto relacional ya implementado, se detallar� cada paso de este proceso. Finalmente se realizar� una comparaci�n de los resultados
obtenidos seg�n los criterios definidos y se presentar�n las ventajas y
desventajas de cada implementaci�n. Por �ltimo en las conclusiones se dar�n las recomendaciones para la
elecci�n del tipo de base de datos seg�n la aplicaci�n a desarrollar, el
cumplimiento de los objetivos del trabajo de titulaci�n, todos los contratiempos
que afectaron la planificaci�n inicial del trabajo, la experiencia personal por
parte de la memorista y el trabajo a futuro.
